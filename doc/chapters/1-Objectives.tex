\section{Objectives}
The rapid proliferation of drone technology has introduced remarkable advancements across various industries, from aerial photography to search and rescue operations. However, this growth has also highlighted significant concerns regarding drone security and privacy. A notable contribution to this field is the study presented in the paper "\textit{Drone Security and the Mysterious Case of DJI’s DroneID}", which explores the vulnerabilities and mechanisms of DJI's DroneID system, a feature designed to enable identification and tracking of drones.

This lab project builds upon the foundational concepts outlined in the aforementioned paper. Our objective is to reproduce certain experiments from the study, focusing on evaluating the security mechanisms of a commercially available, low-cost Chinese drone (E88). By conducting this analysis, we aim to understand how drone security principles can be applied or adapted in less sophisticated, budget-friendly drone models, which are becoming increasingly popular among hobbyists and small-scale operators.

Through this investigation, we not only aim to illuminate potential vulnerabilities in these systems and assess the feasibility of applying similar identification and tracking methods but also strive to approach the broader realm of Internet of Things (IoT) devices. This experiment provides an opportunity to deepen our understanding of how to conduct research and effectively interface with such devices, bridging theoretical insights with practical experimentation. By engaging with IoT concepts in this context, we aim to build foundational skills for tackling security and interaction challenges in this rapidly evolving technological landscape.

In this report, we will outline the process that led us to uncover the communication methods used by the drone, describe the connection tests we conducted, and detail the development of a custom fuzzer. This fuzzer was designed to identify potential vulnerabilities or inconsistencies in the drone's communication protocols, offering insights into its operational security and resilience.